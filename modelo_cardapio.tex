\documentclass[12pt, a4paper]{article}

\usepackage{fontspec}
\usepackage[brazil]{babel}
\usepackage{graphicx}
\usepackage[space]{grffile}
\usepackage{amsmath}
\usepackage[svgnames]{xcolor}
\usepackage{geometry}
\usepackage{enumitem}
\usepackage{tasks}
\usepackage{longtable}
\usepackage{ragged2e}

\setmainfont{Times New Roman}
\geometry{a4paper, top=1.5cm, bottom=1.5cm, left=2cm, right=2cm}

\pagestyle{empty}

\begin{document}
\begin{center}
    {\Huge \textbf{Cardápio de Questões}}
\end{center}
%\vspace{1cm}

<<% for questao in questoes %>>

% --- Cabeçalho do Tema e Disciplina ---
<<% if loop.first or questao.tema != loop.previtem.tema or questao.disciplina_id != loop.previtem.disciplina_id %>>
    <<% if not loop.first %>>\vspace{1.5cm}<<% endif %>>
    \begin{center}
        \fcolorbox{gray}{lightgray}{%
            \parbox{\dimexpr\textwidth-2\fboxsep-2\fboxrule\relax}{%
                \centering
                \textbf{\large << obter_disciplina_nome_por_id(questao.disciplina_id) >>} \\
                \vspace{2mm}
                \textbf{\Large << questao.tema >>}
            }%
        }%
    \end{center}
    %\vspace{0.5cm}
<<% endif %>>

% --- INÍCIO DA MUDANÇA: MINIPAGE E FBOX ---
% Cada questão agora é um bloco dentro de uma caixa
\par\noindent
\fbox{\begin{minipage}{0.96\textwidth}
\vspace{0.2cm}
% --- FIM DA MUDANÇA ---

% --- Corpo da Questão ---
\par\noindent\textbf{ID da Questão: << questao.id_base >>} 
<<% if questao.ativa %>>
    \textbf{(Ativa)}
<<% else %>>
    \textcolor{red}{\textbf{(Inativa)}}
<<% endif %>>

% --- INÍCIO DA MUDANÇA: TEXTO JUSTIFICADO ---
% Enunciado agora dentro de um ambiente 'justify'
\begin{justify}
<< questao.enunciado >>
\end{justify}
% --- FIM DA MUDANÇA ---

% Imagem
<<% if questao.imagem %>>
    \par\vspace{0.1cm}
    \par\noindent\centering
    \includegraphics[width=<<questao.imagemLarguraPercentual / 100.0>>\textwidth]{<<questao.imagem>>}
    \par\vspace{0.1cm}
<<% endif %>>

% --- LÓGICA CONDICIONAL PARA O FORMATO DA QUESTÃO ---
<<% if questao.formato_questao == 'Múltipla Escolha' and questao.alternativas %>>
    <<% if questao.is_multi_valor %>>
        % Layout Horizontal (com ajuste para warnings)
            \begin{tasks}[label=\textbf{(\Alph*)}, label-width=1em, label-offset=0.8em](5)
            <<% for letra, texto in questao.alternativas.items()|sort %>>
                \task 
                    <<% if texto == questao.resposta_valor %>>
                        \textbf{<< texto >>}
                    <<% else %>>
                        << texto >>
                    <<% endif %>>
            <<% endfor %>>
            \end{tasks}
    <<% else %>>
       % Layout Vertical
        \begin{enumerate}[label=\textbf{(\Alph*)}, wide, labelwidth=!, labelindent=0pt, leftmargin=*, topsep=0.5cm, itemsep=0.3cm]
        <<% for letra, texto in questao.alternativas.items()|sort %>>
            \item 
                <<% if texto == questao.resposta_valor %>>
                    \textbf{<< texto >>}
                <<% else %>>
                    << texto >>
                <<% endif %>>
        <<% endfor %>>
        \end{enumerate}
    <<% endif %>>
<<% elif questao.formato_questao == 'Verdadeiro ou Falso' %>>
    \vspace{0.5cm}
    \noindent ( \quad ) Verdadeiro \hspace{2cm} ( \quad ) Falso
<<% elif questao.formato_questao == 'Discursiva' %>>
    \vspace{5cm}
<<% endif %>>

%\par\vspace{0.5cm}

% --- INÍCIO DA MUDANÇA: FIM DO MINIPAGE ---
\end{minipage}}
\vspace{2mm} % Espaço entre as caixas das questões
% --- FIM DA MUDANÇA ---

<<% endfor %>>

\end{document}