% modelo_gabarito.tex
\documentclass[12pt, a4paper]{article}

\usepackage{fontspec}
\usepackage[brazil]{babel}
\usepackage{graphicx}
\usepackage{geometry}
\usepackage[svgnames,table]{xcolor} % Garante compatibilidade de cor com tabelas
\usepackage{fancyhdr}
\usepackage{array}
\usepackage{colortbl}

\setmainfont{Times New Roman}
\geometry{a4paper, top=2.5cm, bottom=2.5cm, left=2cm, right=2cm}
\setlength{\headheight}{50pt}
\pagestyle{fancy}
\fancyhf{}
\fancyhead[L]{\includegraphics[height=1.5cm]{logo-iff.png}}
\fancyhead[C]{\textbf{\Large GABARITO OFICIAL}}
\fancyfoot[C]{Página \thepage}
\renewcommand{\headrulewidth}{0.4pt}
\renewcommand{\footrulewidth}{0.4pt}

\begin{document}

<<% for versao in versoes %>>
    <<% if not loop.first %>>
        \bigskip
        \bigskip
    <<% endif %>>

    \section*{\centering << versao.versao >>}
    
    % Centraliza a tabela na página
    \begin{center}
    % Tabela trocada para 'tabular' para ser mais compacta e colunas reajustadas
    \begin{tabular}{|l|c|c|c|}
        \hline
        \rowcolor{lightgray} % Comando de cor para a linha do cabeçalho
        \textbf{Tema} & \textbf{ID} & \textbf{Questão} & \textbf{Resposta} \\
        \hline

        <<% for item in versao.itens %>>
            << item.tema >> & << item.id >> & << loop.index >> & \textbf{<< item.resposta >>} \\
            \hline
        <<% endfor %>>
    \end{tabular}
    \end{center}

<<% endfor %>>

\end{document}