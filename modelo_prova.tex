% modelo_prova.tex (versão corrigida e limpa)
\documentclass[12pt, a4paper]{article}

\usepackage{fontspec}
\usepackage[brazil]{babel}
\usepackage{graphicx}
\usepackage[space]{grffile}
\usepackage{amsmath}
\usepackage{geometry} 
\usepackage[svgnames]{xcolor}
\usepackage{fancyhdr}
\usepackage{lastpage}
\usepackage{enumitem}
\usepackage{tabularx}
\usepackage{amsfonts}
\usepackage{amssymb}
\usepackage{ifthen}
\usepackage{tasks}

\setmainfont{Times New Roman}

% --- Geometria da Página ---
% Definimos a altura do cabeçalho primeiro para o geometry calcular o layout corretamente.
\setlength{\headheight}{50pt} 
\geometry{a4paper, top=2.5cm, bottom=2.5cm, left=2cm, right=2cm, includehead, includefoot}

% --- Configuração do Cabeçalho e Rodapé ---
\pagestyle{fancy}
\fancyhf{} % Limpa todos os campos
\fancyhead[L]{\includegraphics[height=1.5cm]{logo-iff.png}}

% --- CABEÇALHO CORRIGIDO ---
% Usamos um tabular para criar as duas linhas sem erro.
\fancyhead[R]{%
    \begin{tabular}{@{}r@{}} % O @{} remove o padding da tabela
        \textbf{VERSÃO} \\ 
        \textbf{<<versao>>}
    \end{tabular}%
}

% --- Rodapé que você ajustou ---
\fancyfoot[L]{\small Prof. <<nomeProfessor>>}
\fancyfoot[C]{\small <<instituicao>>} 
\fancyfoot[R]{\thepage}

\renewcommand{\headrulewidth}{0.4pt}
\renewcommand{\footrulewidth}{0.4pt}

% --- Comando para o Título da Questão ---
\newcommand{\questiontitle}[2]{%
    \par\noindent
    \fcolorbox{gray}{lightgray}{%
        \parbox{\dimexpr\textwidth-2\fboxsep-2\fboxrule\relax}{%
            \hfill\textbf{QUESTÃO #1}\hfill
            \ifthenelse{\equal{#2}{}}{}{%
                \textbf{(valor: #2 pts)}%
            }%
        }%
    }%
    \par\vspace{0.4cm}%
}


% #################### INÍCIO DO DOCUMENTO ####################
\begin{document}
\thispagestyle{empty}

% (Capa da prova)
\begin{center}
    \fcolorbox{black}{black}{%
        \begin{minipage}[c][2.5cm][c]{0.7\textwidth}
            \centering \color{white}
            {\sffamily\bfseries\fontsize{22}{26}\selectfont CCTECC} \\[0.5cm]
            {\small Coordenação do Curso Técnico em Eletrotécnica - Campus Centro}
        \end{minipage}%
    }
\end{center}
\vspace*{0.5cm}
\begin{center}
    \fcolorbox{black}{black}{%
        \begin{minipage}[c][2cm][c]{0.8\textwidth}
            \centering \color{white} \fontsize{24}{28}\selectfont \textbf{AVALIAÇÃO BIMESTRAL}
        \end{minipage}%
    }\fcolorbox{black}{DarkGray}{%
        \begin{minipage}[c][2cm][c]{0.12\textwidth}
            \centering \color{white} \fontsize{36}{40}\selectfont \textbf{<<bimestre>>}
        \end{minipage}%
    }
\end{center}
\vspace*{0.5cm}
\noindent\fcolorbox{gray}{lightgray}{\parbox{\dimexpr\textwidth-2\fboxsep-2\fboxrule\relax}{\centering\textbf{\Large <<nomeDisciplina|upper>> - <<tipoExame|upper>>}}}
\vspace*{0.5cm}
\noindent\fcolorbox{gray}{lightgray}{\parbox{\dimexpr\textwidth-2\fboxsep-2\fboxrule\relax}{\centering\textbf{LEIA COM ATENÇÃO AS INSTRUÇÕES ABAIXO.}}}
\vspace*{0.5cm}
\begin{enumerate}[label=\bfseries\arabic*. , wide, labelwidth=!, labelindent=0pt, leftmargin=*]
    \item Você está recendo o seguinte material:
        \begin{enumerate}[label=\alph*)]
            \item 1 caderno de questões, contendo: \\
            \begin{center}
                \begin{tabular}{|c|c|c|} \hline
                    \textbf{Quantidade de questões} & \textbf{Valor de cada questão} & \textbf{Valor Total da Prova} \\ \hline
                    <<numeroQuestoes>> & <<valorPorQuestao>> & <<valorTotalProva>> \\ \hline
                \end{tabular}
            \end{center}
            \item 1 cartão de respostas, destinado às respostas das questões de múltipla escolha.
        \end{enumerate}
    \item Verifique se este material está completo. Caso contrário, notifique imediatamente ao professor.
    \item Observe no Cartão-Resposta as instruções sobre a marcação das respostas.
    \item Tenha muito cuidado com o Cartão-Resposta, para não dobrar, amassar ou manchar.
    \item Esta prova é individual e sem consulta.
\end{enumerate}
\vspace{0.5cm}
\noindent\textbf{Nome:} \rule{0.8\textwidth}{0.4pt} \\
\vspace{0.5cm}
\noindent\textbf{Data:} \rule{0.4\textwidth}{0.4pt}
\newpage

% ========= CORPO DA PROVA (QUESTÕES) ===========
<<% for questao in questoes %>>
% --- Início da "caixa" que impede a quebra da questão ---
    \begin{minipage}{\linewidth} 
    \questiontitle{<<loop.index>>}{<<questao.valor>>}
    
    <<questao.enunciado>>

    <<% if questao.imagem %>>
        \par\vspace{0.1cm}
        \par\noindent\centering
        \includegraphics[width=<<questao.imagemLarguraPercentual / 100.0>>\textwidth]{<<questao.imagem>>}
        \par\vspace{0.1cm}
    <<% endif %>>

    <<% if questao.formato_questao == 'Múltipla Escolha' %>>
        <<% if questao.is_multi_valor %>>
            % --- LAYOUT HORIZONTAL (Apenas para Múltiplos Valores) ---
            \begin{tasks}[label=\textbf{(\Alph*)}, label-offset=0.8em](5)
            <<% for letra, texto in questao.alternativas.items()|sort %>>
                \task << texto >>
            <<% endfor %>>
            \end{tasks}
        <<% else %>>
            % --- LAYOUT VERTICAL (Para Múltipla Escolha comum) ---
            \begin{enumerate}[label=\textbf{(\Alph*)}, wide, labelwidth=!, labelindent=0pt, leftmargin=*, topsep=0.5cm, itemsep=0.3cm]
            <<% for letra, texto in questao.alternativas.items()|sort %>>
                \item << texto >>
            <<% endfor %>>
            \end{enumerate}
        <<% endif %>>
    <<% elif questao.formato_questao == 'Verdadeiro ou Falso' %>>
        \vspace{0.5cm}
        \noindent ( \quad ) Verdadeiro \hspace{2cm} ( \quad ) Falso
    <<% elif questao.formato_questao == 'Discursiva' %>>
        \vspace{5cm}
    <<% endif %>>
    
    \vspace{1cm}
    \end{minipage} 
    % --- Fim da "caixa" ---

<<% endfor %>>
    
\end{document}